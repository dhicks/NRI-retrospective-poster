\documentclass[a0paper, landscape]{baposter}

\usepackage{hyperref}
\usepackage{lipsum}

\definecolor{boxcolor}{RGB}{0,215,230}

\begin{document}
\typeout{Poster rendering started}

\begin{poster}{
	grid = false, 
	columns = 4,
	eyecatcher = true, 
%	borderColor=bordercol,
%	headerColorOne=headercol1,
%	headerColorTwo=headercol2,
%	headerFontColor=headerfontcol,
%	boxColorOne=boxcolor,
%	headershape=roundedright,
%	headerfont=\Large\sf\bf,
	headershape = rounded,
	headerborder = open,
	headerColorOne = blue, 
	headerColorTwo = blue,
	headerFontColor = white,
	textborder=rectangle,
	boxColorOne = white,
	borderColor = black,
	background=none,
%	headerborder=open,
%  boxshade=plain
}
%% Eye catcher
{
	Eye Catcher, empty if option eyecatcher=false - unused
}
%% Title
{
	\textbf{The National Robotics Initiative}\\
	\normalsize A Five-Year Retrospective
}
%% Authors
{
	Daniel J. Hicks; Reid Simmons\\
	\small \tt hicks.daniel.j@gmail.com; [email]
}
%% Logo
{
	% NSF logo and such
}

\headerbox{Introduction}{name=intro, column = 0, row = 0}{
	\lipsum[1]
}

\headerbox{Data Sources}{name = data, column = 0, below = intro}{
	Data for this poster were retrieved from NSF's public awards database, available at \url{www.nsf.gov/awardsearch/advancedSearch.jsp}.  The dataset includes start and end date, personnel, institutional affiliation, and short descriptions for every project funded by program element code 8013, NRI.  
}

\headerbox{By the Numbers}{name = stats, column = 0, below = data}{
	To date, NRI has funded over
	\begin{itemize}
	\item 165 distinct projects, involving
	\item 306 PIs and Co-PIs from
	\item 97 research universities and other organizations.  
	\end{itemize}
}

\headerbox{Projects}{name = projects, column = 1}{
	This strip plot shows the start and (anticipated) end date for each of the NRI projects, sorted by start date.
	\includegraphics[width = \linewidth]{../stripplot.pdf}
}

\headerbox{Collaboration Networks}{name = networks, column = 2}{
	NRI has fostered connections between individual researchers and their home institutions.  In the first collaboration network, individual PIs and Co-PIs (nodes) are connected by their institutional affiliations (blue edges) and NRI-funded projects (red edges).  In the second collaboration network, universities and other organizations (nodes) are connected by their participation in NRI-funded projects (edges).  Both plots show the largest connected component in the respective networks.  These components include 60\% of NRI-funded PIs and Co-PIs (first network) and 45\% of NRI-participating organizations (second network).  
	
	\begin{center}
		\includegraphics[width = .75\linewidth]{../individuals.pdf}\\
		\includegraphics[width = .75\linewidth]{../organizations.pdf}
	\end{center}
}

\headerbox{Text Analysis}{name = text, column = 3}{
%	\lipsum[2]
	Three text analysis methods were used to attempt to identify common themes or clusters of NRI projects: topic modeling with Latent Dirichlet Allocation [LDA] [cite], principal components analysis over terms with high term frequency-inverse document frequency [tf-idf] [cite], and hierarchical clustering using Jaccard distances over high-tf-idf terms [cite].  
	
	[cite] describe a method to evaluate the stability of LDA over a given corpus.  In the plot below, for each value of $k$, a LDA model with $k$ topics is fit to 25 subsets comprising 132 project descriptions.  These subset models are then compared to a reference model using all 165 project descriptions.  Agreement scores range from 0 to 1, with 1 indicating perfect agreement between the subset model and the reference model.  

	\includegraphics[width = \linewidth]{../lda.pdf}
	
	For values of $k$ from 2 to 12, agreement was uniformly below .5 and frequently below .2, indicating that LDA produces unstable results when applied to short project descriptions.  The two methods based on tf-idf also produced low-quality clustering.  Text analysis is likely to be more informative with longer source texts, such as project descriptions included with proposal submissions.  
}

\headerbox{References}{column = 1, below = projects}{
	\begin{itemize}
		\item ref1
		\item ref2
		\item ref3
	\end{itemize}
}

%\headerbox{}{name = logo, column = 3, below = text}{
%	\begin{center}
%	\includegraphics[height = 2in]{fake_logo.jpg}
%	\end{center}
%}

\end{poster}
\end{document}